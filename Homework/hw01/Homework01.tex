\documentclass[12pt]{article}
\usepackage[paper=letterpaper, left=1in, right=1in, top=1in, bottom=1in]{geometry}

\usepackage[parfill]{parskip}
\usepackage{amssymb}
\usepackage{amsmath}
\pagestyle{empty}

\begin{document}

\begin{center}
{\large CS 291}\\
Homework 1
\end{center}

\begin{flushright}
Jingbo Wang\\
jw6347@truman.edu
\end{flushright}

\textbf{Section 6.2, problem 7.e}  Use Quine's method to show that each wff is a contingency.
$(A \rightarrow B) \lor ((C \rightarrow \neg B) \land \neg C)$.

\emph{Answer:} First we compute the wff $W(A/\text{true})$.

\begin{center}
\begin{eqnarray*}
W(A/\text{true}) & \equiv & (\text{true} \rightarrow B) \lor ((C
\rightarrow \neg B) \land \neg C) \\ 
& \equiv & B \lor ((C \rightarrow \neg B) \land \neg C) \\ 
\end{eqnarray*}
\end{center}

Let $ X = B \lor ((C \rightarrow \neg B) \land \neg C).$ Then we have:

\begin{center}
\begin{eqnarray*}
X(B/\text{true}) & \equiv & \text{true} \lor ((C \rightarrow 
\text{false}) \land \neg C) \\
& \equiv & \text{true} \lor (\neg C \land \neg C) \\
& \equiv & \text{true} \lor \neg C \\ 
& \equiv & \text{true} \\
X(B/\text{false}) & \equiv & \text{false} \lor ((C \rightarrow 
\text{true}) \land \neg C) \\
& \equiv & \text{false} \lor (\text{true} \land \neg C) \\
& \equiv & \text{false} \lor \neg C \\
\end{eqnarray*}
\end{center}

Let $ Y = \text{false} \lor \neg C.$ Then we have:

\begin{center}
\begin{eqnarray*}
Y(C/\text{true}) & \equiv & \text{false} \lor \text{false} \\
& \equiv & \text{false} \\
Y(C/\text{false}) & \equiv & \text{false} \lor \text{true} \\
& \equiv & \text{true} \\
\end{eqnarray*}
\end{center}

Therefore $W(A/\text{true})$ is not a tautology.  Next we check $W(A/\text{false})$.

\begin{center}
\begin{eqnarray*}
W(A/\text{false}) & \equiv (\text{false} \rightarrow B) \lor ((C
\rightarrow \neg B) \land \neg C) \\ 
& \equiv \text{true} \lor ((C \rightarrow \neg B) \land \neg C)  \\
\end{eqnarray*}
\end{center}

Let $ X = \text{true} \lor ((C \rightarrow \neg B) \land \neg C).$ Then we have:

\begin{center}
\begin{eqnarray*}
X(C/\text{true}) & \equiv & \text{true} \lor (\text{true} 
\rightarrow \neg B) \land \text{false}) \\
& \equiv & \text{true} \lor (\neg B \land \text{false}) \\
& \equiv & \text{true} \lor \text{false} \\
& \equiv & \text{true} \\
X(C/\text{false}) & \equiv & \text{true} \lor (\text{false} 
\rightarrow \neg B) \land \text{true}) \\
& \equiv & \text{true} \lor (\text{true} \land \text{true}) \\
& \equiv & \text{true} \lor \text{true} \\
& \equiv & \text{true} \\
\end{eqnarray*}
\end{center}

This produces a contradiction. Thus the wff is a contingency.

\textbf{Section 6.2, problem 8.f} Use Quine's method to show that each wff is a tautology. \\
$(A\rightarrow B) \rightarrow (C \lor A \rightarrow C \lor B)$

\emph{Answer:} First we compute the wff $W(A/\text{true})$.

\begin{center}
\begin{eqnarray*}
W(A/\text{true}) & \equiv & (\text{true}\rightarrow B) 
\rightarrow (C \lor \text{true} \rightarrow C \lor B) \\ 
& \equiv & B \rightarrow (true \rightarrow  C \lor B) \\ 
& \equiv & B \rightarrow (C \lor B)
\end{eqnarray*}
\end{center}

Let $ X = B \rightarrow (C \lor B).$ Then we have:

\begin{center}
\begin{eqnarray*}
X(B/\text{true}) & \equiv & \text{true} \rightarrow (C \lor \text{true}) \\
& \equiv & \text{true} \rightarrow \text{true} \\
& \equiv & \text{true} \\
X(B/\text{false}) & \equiv & \text{false} \rightarrow (C \lor \text{fale}) \\
& \equiv & \text{false} \rightarrow C \\
& \equiv & \text{true} \\
\end{eqnarray*}
\end{center}

Therefore $W(A/\text{true})$ is a tautology.  Next we check $W(A/\text{false})$.

\begin{center}
\begin{eqnarray*}
W(A/\text{false}) & \equiv & (\text{false}\rightarrow B) 
\rightarrow (C \lor \text{false} \rightarrow C \lor B) \\ 
& \equiv & \text{true} \rightarrow (C \rightarrow C \lor B) \\
& \equiv & C \rightarrow C \lor B 
\end{eqnarray*}
\end{center}

Let $ X = \text{true} \rightarrow (C \rightarrow C \lor B).$ Then we have:

\begin{center}
\begin{eqnarray*}
X(C/\text{true}) & \equiv & \text{true} \rightarrow \text{true} \lor B \\ 
& \equiv & \text{true} \\
X(C/\text{false})
& \equiv & \text{false} \rightarrow \text{false} \lor B \\ 
& \equiv & \text{false} \rightarrow \text{false} \\
& \equiv & \text{true} \\
\end{eqnarray*}
\end{center}

Thus the wff is a tautology.

\textbf{Section 6.2, problem 9.b} Verify each of the following equivalences by writing an equivalence proof.
That is, start on one side and use known equivalences to the other side. \\
$(A\land B) \rightarrow C \equiv (A \rightarrow C) \lor (B \rightarrow C)$.

\emph{Answer:} We start with the left hand side and derive the right,
using the equivalences given in Table 6.2.5 on page 475.

\begin{align*}
(A \land B) \rightarrow C
  & \equiv \neg (A \land B) \lor C                       && \text{1st conversion} \\
  & \equiv (\neg A \lor \neg B) \lor C                  && \text{De Morgan's Laws}\\
  & \equiv (\neg A \lor C) \lor (\neg B \lor C)         && \text{distribute property} \\
  & \equiv (A \rightarrow C) \lor (B \rightarrow C)     && \text{1st conversion}
\end{align*}

\textbf{Section 6.2, problem 11.e} Use equivalences to transform each of the following wffs into a DNF. \\
$P \rightarrow Q \land R$.
\begin{align*}
P \rightarrow Q \land R
 & \equiv \neg P \lor Q \land R \\
 & \equiv \neg P \lor (Q \land R) \\
\end{align*}

\textbf{Section 6.2, problem 12.d} Use equivalences to transform each of the following wffs into a CNF. \\
$(P \lor Q) \land R$.
\begin{align*}
(P \lor Q) \land R
 & \equiv (P \lor Q) \land R \\
\end{align*}


\end{document}