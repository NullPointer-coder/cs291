\documentclass[12pt]{article}
\usepackage[paper=letterpaper, left=1in, right=1in, top=1in, bottom=1in]{geometry}

\usepackage[parfill]{parskip}
\usepackage{amssymb}
\usepackage{amsmath}
\pagestyle{empty}

\begin{document}

\begin{center}
{\large CS 291}\\
Homework 3
\end{center}

\begin{flushright}
Jingbo Wang \\
jw6347@truman.edu
\end{flushright}

\textbf{Section 7.1, problem 5.b}  For each of the following wffs, 
label each occurrence of the variable as either bound free.

$\forall y \enspace q(y) \land \neg p(x,y)$.

\emph{Answer:} 
\begin{center}
The one occurrences of x, left to right, is free. \\
The three occurrences of y, left to right, are bound, bound, and free.\\
\end{center}


\textbf{Section 7.1, problem 5.c}  For each of the following wffs, 
label each occurrence of the variable as either bound free.

$\neg q(x,y) \lor \exists x \enspace p(x,y)$.

\emph{Answer:} 
\begin{center}
The three occurrences of x, left to right, is free, bound, and bound. \\
The two occurrences of y, left to right, both are free. \\
\end{center}


\textbf{Section 7.1, problem 7.e}  Let isFatherOf(\emph{x}, \emph{y}) be “\emph{x} is the father of \emph{y},” 
where the domain is the set of all people now living or who have lived. 
Find the truth value for each of the following wffs.

$\exists x \enspace \exists y \enspace isFatherOf(x, y)$

\emph{Answer:} 
\begin{center}
$\exists x \enspace \exists y \enspace isFatherOf(x, y)$ means some \emph{x} are the father of some \emph{y}, so it is \text{True} \\
\end{center}


\textbf{Section 7.1, problem 7.f}  Let isFatherOf(\emph{x}, \emph{y}) be “\emph{x} is the father of 
\emph{y},” 
where the domain is the set of all people now living or who have lived. 
Find the truth value for each of the following wffs.

$\exists y \enspace \exists x \enspace isFatherOf(x, y)$

\emph{Answer:} 
\begin{center}
$\exists y \enspace \exists x \enspace isFatherOf(x, y)$ means some \emph{y} are the son of some \emph{x}, so it is \text{True} \\
\end{center}


\textbf{Section 7.1, problem 10.a}  Let \emph{B}(\emph{x}) mean \emph{x} is a bird, 
let \emph{W}(\emph{x}) mean \emph{x} is a worm, and let \emph{E}(\emph{x}, \emph{y}) mean \emph{x} eats \emph{y}. 
Find an English sentence to describe each of the following statements.

$\forall x \enspace \forall y \enspace (B(x) \land W(y) \rightarrow E(x, y))$.

\emph{Answer:} 
\begin{center}
 Every bird eats every worm. \\
\end{center}


\textbf{Section 7.1, problem 10.b}  Let \emph{B}(\emph{x}) mean \emph{x} is a bird, 
let \emph{W}(\emph{x}) mean \emph{x} is a worm, and let \emph{E}(\emph{x}, \emph{y}) mean \emph{x} eats \emph{y}. 
Find an English sentence to describe each of the following statements.

$\forall x \enspace \forall y \enspace (E(x, y)\rightarrow B(x) \land W(y))$.

\emph{Answer:} 
\begin{center}
If every \emph{x} eats every \emph{y}, then \emph{x} is bird and \emph{y} is worm.\\
\end{center}


\textbf{Section 7.2, problem 4.c} Use equivalences to construct 
a prenex conjunctive normal form for each of the following wffs.

$\forall x \enspace \exists y \enspace p(x,y) \rightarrow \exists y 
\enspace \forall x \enspace p(x,y)$.

\emph{Answer:} We know that a prenex normal form is called a prenex conjunctive normal form 
if has the form $Q_1x_1...Q_m x_m[D_1 \land ...\land D_k]$.
\begin{align*}
\forall x \enspace \exists y \enspace p(x,y) \rightarrow \exists y 
\enspace \forall x \enspace p(x,y) 
    & \equiv \forall x \enspace \exists y \enspace p(x,y) \rightarrow 
      \exists z \enspace \forall w \enspace p(w,z)          && \text{(rename)} \\
    & \equiv \neg ( \forall x \enspace \exists y \enspace p(x,y)) \lor 
      \exists z \enspace \forall w \enspace p(w,z)          && \text{(remove $\rightarrow$)} \\ 
    & \equiv \neg \forall x \enspace \exists y \enspace \neg p(x,y) \lor 
      \exists z \enspace \forall w \enspace p(w,z)          && \text{(De Morgan's Laws)} \\
    & \equiv \exists x \enspace \forall y \enspace \neg p(x,y) \lor 
      \exists z \enspace \forall w \enspace p(w,z)         && \text{(1)} \\
    & \equiv \exists x \enspace \forall y \enspace \exists z \enspace 
      \forall w \enspace (\neg p(x,y) \lor p(w,z) )         && \text{(7b)} \\
\end{align*}


\textbf{Section 7.2, problem 4.d} Use equivalences to construct 
a prenex conjunctive normal form for each of the following wffs.

$\forall x \enspace (p(x,f(x)) \rightarrow p(x,y))$.

\emph{Answer:} We know that a prenex normal form is called a prenex conjunctive normal form 
if has the form $Q_1x_1...Q_m x_m[D_1 \land ...\land D_k]$.
\begin{align*}
\forall x \enspace (p(x,f(x)) \rightarrow p(x,y))
    & \equiv \forall x \enspace (\neg p(x,f(x)) \lor p(x,y))     && \text{(remove $\rightarrow$)} \\
\end{align*}


\textbf{Section 7.2, problem 5.d} Use equivalences to construct 
a prenex conjunctive normal form for each of the following wffs.

$\forall x \enspace (p(x,f(x)) \rightarrow p(x,y))$.

\emph{Answer:}We know that a prenex normal form is called a prenex disjunctive normal form 
if has the form $Q_1x_1...Q_m x_m[D_1 \lor ...\lor D_k]$.
\begin{align*}
\forall x \enspace (p(x,f(x)) \rightarrow p(x,y))
    & \equiv \forall x \enspace (\neg p(x,f(x)) \lor p(x,y))    && \text{(remove $\rightarrow$)} \\
\end{align*}


\textbf{Section 7.2, problem 5.e}  Use equivalences to construct 
a prenex conjunctive normal form for each of the following wffs.

$\forall x \enspace \forall y \enspace (p(x,y) \rightarrow \exists z \enspace (p(x,z) \land p(y,z)))$.

\emph{Answer:}We know that a prenex normal form is called a prenex disjunctive normal form 
if has the form $Q_1x_1...Q_m x_m[D_1 \lor ...\lor D_k]$.
\begin{align*}
\forall x \enspace \forall y \enspace (p(x,y) \rightarrow \exists z \enspace (p(x,z) \land p(y,z)))
    & \equiv \forall x \enspace \forall y \enspace (\neg p(x,y) \lor \exists z \\
    & (p(x,z) \land p(y,z)))    && \text{(remove $\rightarrow$)} \\
    & \equiv \forall x \enspace \forall y \enspace \exists z \enspace (\neg p(x,y) \\
    & \lor (p(x,z) \land p(y,z)))                       && \text{(7b)} \\
\end{align*}


\textbf{Section 7.2, problem 8.b}  Formalize each of the following statements, 
where \emph{B}(\emph{x}) means \emph{x} is a bird, \emph{W}(\emph{x}) means \emph{x} is a worm, and 
\emph{E}(\emph{x}, \emph{y}) means \emph{x} eats \emph{y}.

Some birds eat worms.

\emph{Answer:}
\begin{center}
$\exists x(B(x) \rightarrow \exists y (W(y) \land E(x,y)))$    
\end{center}

\textbf{Section 7.2, problem 8.c}  Formalize each of the following statements, 
where \emph{B}(\emph{x}) means \emph{x} is a bird, \emph{W}(\emph{x}) means \emph{x} is a worm, and 
\emph{E}(\emph{x}, \emph{y}) means \emph{x} eats \emph{y}.

Only birds eat worms.

\emph{Answer:} 
\begin{center}
$\forall x \enspace \forall y \enspace (W(x) \land E(y,x) \rightarrow B(y))$\\
\end{center}

\textbf{Section 7.2, problem 8.d}  Formalize each of the following statements, 
where \emph{B}(\emph{x}) means \emph{x} is a bird, \emph{W}(\emph{x}) means \emph{x} is a worm, and 
\emph{E}(\emph{x}, \emph{y}) means \emph{x} eats \emph{y}.

Not all birds eat worms

\emph{Answer:}
\begin{center}
$\neg \forall x(B(x) \rightarrow \exists y (W(y) \land E(x,y)))$    
\end{center}

\end{document}